%****************************************************************************
%** Copyright 2002 by Lukas Ruf, ruf@topsy.net
%** Information is provided under the terms of the
%** GNU Free Documentation License http://www.gnu.org/copyleft/fdl.html
%** Fairness: Cite the source of information, visit http://www.topsy.net
%****************************************************************************

%Example structure for an introduction
%****************************************************************************
%** Copyright 2002, 2003 by Lukas Ruf, <ruf@topsy.net>
%** Information is provided under the terms of the
%** GNU Free Documentation License <http://www.gnu.org/copyleft/fdl.html>
%** Fairness: Cite the source of information, visit <http://www.topsy.net>
%****************************************************************************

\chapter{\label{introduction}Introduction}
The border gateway protocol (BGP) is the glue that holds the Internet together. It allows autonomous networks to exchange information about reachability of prefixes without revealing their own network infrastructure. \\
Remote disruptions in BGP can lead to convergence times of multiple minutes. This long convergence time is caused by the way BGP updates are propagated through the internet. The convergence time is lower bounded by the time it takes for all BGP withdrawal updates to be received. \\
Software defined networking allows network operators to have more control over the network routing. It is also a technology that can help solve some of the internet's problems including long BGP convergence times. \\
In this work I will try to improve the convergence time of the industrial scale internet exchange point (iSDX) using the Swift framework, both rely on SDN switches to enable their functionality. 

\rb{In the following, we first provide some background on BGP, iSDX and Swift in \ref{chapter2}. Then, we describe the challenge of implementing Swift in iSDX in \ref{chapter3}. In \ref{chapter4} we evaluate the system both in terms of convergence time and introduced overhead and discuss the results in \ref{chapter5}.}

